\documentclass{article}
\usepackage[a4paper,
            bindingoffset=0.2in,
            left=0.25in,
            right=0.25in,
            top=0.25in,
            bottom=0.25in,
            footskip=0.25in]{geometry}
\usepackage{amsmath}
\usepackage{amssymb}
\usepackage{graphicx}
\usepackage{listings}
\usepackage{color}

\definecolor{mygreen}{rgb}{0,0.6,0}
\definecolor{mygray}{rgb}{0.5,0.5,0.5}
\definecolor{mymauve}{rgb}{0.58,0,0.82}

\lstset{ 
  backgroundcolor=\color{white},   % choose the background color; you must add \usepackage{color} or \usepackage{xcolor}; should come as last argument
  basicstyle=\footnotesize,        % the size of the fonts that are used for the code
  breakatwhitespace=false,         % sets if automatic breaks should only happen at whitespace
  breaklines=true,                 % sets automatic line breaking
  captionpos=b,                    % sets the caption-position to bottom
  commentstyle=\color{mygreen},    % comment style
  deletekeywords={...},            % if you want to delete keywords from the given language
  escapeinside={\%*}{*)},          % if you want to add LaTeX within your code
  extendedchars=true,              % lets you use non-ASCII characters; for 8-bits encodings only, does not work with UTF-8
  firstnumber=1,                   % start line enumeration with line 1
  frame=single,	                   % adds a frame around the code
  keepspaces=true,                 % keeps spaces in text, useful for keeping indentation of code (possibly needs columns=flexible)
  keywordstyle=\color{blue},       % keyword style
  language=Python,                 % the language of the code
  morekeywords={},                 % if you want to add more keywords to the set
  numbers=left,                    % where to put the line-numbers; possible values are (none, left, right)
  numbersep=5pt,                   % how far the line-numbers are from the code
  numberstyle=\color{black},       % the style that is used for the line-numbers
  showspaces=false,                % show spaces everywhere adding particular underscores; it overrides 'showstringspaces'
  showstringspaces=false,          % underline spaces within strings only
  showtabs=false,                  % show tabs within strings adding particular underscores
  stepnumber=1,                    % the step between two line-numbers. If it's 1, each line will be numbered
  stringstyle=\color{mymauve},     % string literal style
  tabsize=4,	                   % sets default tabsize to 4 spaces
  title=\lstname                   % show the filename of files included with \lstinputlisting; also try caption instead of title
}

\title{My Expirence Animating with Manim}
\author{Michael Xie \\ Jack Miner Public School \\ Grade 8}
\date{\today}

\begin{document}
\maketitle

\section{Introduction}
Manim is a powerful animation engine that allows you to create high-quality animations for mathematical concepts. It is widely used in educational videos and presentations. In this article, I will share my experience using Manim 
to create animations for my class project. I will discuss the challenges I faced, the solutions I found, and the final product I created.

\section{Getting Started}
I already had a Manim environment set up on my computer from a previous project, along with a solid understanding of Python. To begin, I researched my topic, holomorphic dynamics, by reading several articles and watching videos, including 3Blue1Brown's excellent explanation on the subject. This gave me a clear idea of what a Manim animation should look like. 

I started coding by using Matplotlib to create a static plot of the holomorphic function I was working with—the Mandelbrot set. After a few rounds of debugging, I successfully wrote a function that generates an ndarray representing the Mandelbrot set, along with several additional utility functions. These included:

\begin{enumerate}
    \item \texttt{mandelbrot}: Generate a Mandelbrot set image.\\
    This function computes the Mandelbrot set for a given range of real and imaginary values,
    and returns a 2D array representing the fractal image. The computation is optimized by
    skipping points inside the main cardioid and the period-2 bulb.\\
    \textbf{Parameters:}
    \begin{itemize}
        \item \texttt{rmin} (float): The minimum value of the real axis.
        \item \texttt{rmax} (float): The maximum value of the real axis.
        \item \texttt{cmax} (complex): The maximum value of the imaginary axis (only the imaginary part is used).
        \item \texttt{width} (int): The width of the output image in pixels.
        \item \texttt{height} (int): The height of the output image in pixels. If odd, it will be adjusted to the next even number.
        \item \texttt{maxiter} (int): The maximum number of iterations to determine divergence.
    \end{itemize}
    \textbf{Returns:}
    \begin{itemize}
        \item \texttt{numpy.ndarray}: A 2D array of shape (\texttt{height}, \texttt{width}) containing the Mandelbrot set.
        Each value represents the iteration count at which the point diverged,
        or 0 if the point is in the set.
    \end{itemize}
    \item \texttt{mandelbrot\_point\_cloud}: Generate the orbit for a single point. \\
    This function computes the orbit of a single point in the Mandelbrot set and returns
    a NDarray of complex numbers representing the orbit.\\
    \textbf{Parameters:}
    \begin{itemize}
        \item \texttt{c} (complex): The complex number to evaluate in the Mandelbrot set.
        \item \texttt{maxiter} (int): The maximum number of iterations to perform.
    \end{itemize}
    \textbf{Returns:}
    \begin{itemize}
        \item \texttt{numpy.ndarray:} A 2D array of shape (maxiter, 2), where each row contains the real and imaginary 
        parts of the complex number at each iteration.
    \end{itemize}
    \item \texttt{mandelbrot\_from\_center}: Generate a Mandelbrot set image centered at a specific point with a zoom level.\\
    This function computes the Mandelbrot set for a given center point and zoom level, allowing for detailed exploration of the fractal.\\
    \textbf{Parameters:}
    \begin{itemize}
        \item \texttt{centerpoint} (complex): The center of the viewing field.
        \item \texttt{zoom} (float): The zoom level (higher values zoom in).
        \item \texttt{width} (int): The width of the output image in pixels.
        \item \texttt{height} (int): The height of the output image in pixels.
        \item \texttt{maxiter} (int): The maximum number of iterations to determine divergence.
    \end{itemize}
    \textbf{Returns:}
    \begin{itemize}
        \item \texttt{numpy.ndarray}: A 2D array of shape (\texttt{height}, \texttt{width}) containing the Mandelbrot set.
    \end{itemize}
    \item \texttt{multibrot}: Generate a generalized Mandelbrot set with a custom exponent.\\
    This function computes the fractal for a given exponent, creating variations of the Mandelbrot set.\\
    \textbf{Parameters:}
    \begin{itemize}
        \item \texttt{rmin}, \texttt{rmax} (float): The range of the real axis.
        \item \texttt{cmax} (complex): The maximum value of the imaginary axis.
        \item \texttt{width}, \texttt{height} (int): The dimensions of the output image in pixels.
        \item \texttt{maxiter} (int): The maximum number of iterations.
        \item \texttt{exponent} (int): The exponent used in the fractal formula.
    \end{itemize}
    \textbf{Returns:}
    \begin{itemize}
        \item \texttt{numpy.ndarray}: A 2D array representing the fractal.
    \end{itemize}
    \item \texttt{julia}: Generate a Julia set for a given complex parameter.\\
    This function computes the Julia set, a related fractal to the Mandelbrot set, for a specific complex parameter.\\
    \textbf{Parameters:}
    \begin{itemize}
        \item \texttt{c} (complex): The complex parameter for the Julia set.
        \item \texttt{width}, \texttt{height} (int): The dimensions of the output image in pixels.
        \item \texttt{maxiter} (int): The maximum number of iterations.
    \end{itemize}
    \textbf{Returns:}
    \begin{itemize}
        \item \texttt{numpy.ndarray}: A 2D array representing the Julia set.
    \end{itemize}
    \item \texttt{multicorn}: Generate a Multicorn fractal with a custom exponent.\\
    This function computes the Multicorn fractal, a variation of the Mandelbrot set with conjugation.\\
    \textbf{Parameters:}
    \begin{itemize}
        \item \texttt{rmin}, \texttt{rmax} (float): The range of the real axis.
        \item \texttt{cmax} (complex): The maximum value of the imaginary axis.
        \item \texttt{width}, \texttt{height} (int): The dimensions of the output image in pixels.
        \item \texttt{maxiter} (int): The maximum number of iterations.
        \item \texttt{exponent} (int): The exponent used in the fractal formula.
    \end{itemize}
    \textbf{Returns:}
    \begin{itemize}
        \item \texttt{numpy.ndarray}: A 2D array representing the Multicorn fractal.
    \end{itemize}
    \item \texttt{buhdabrot}: Generate a Buhdabrot fractal using a 4D histogram.\\
    This function computes the Buhdabrot fractal by sampling random points and tracking their trajectories in a 4D histogram.\\
    \textbf{Parameters:}
    \begin{itemize}
        \item \texttt{rmin}, \texttt{rmax} (float): The range of the real axis.
        \item \texttt{cmax} (complex): The maximum value of the imaginary axis.
        \item \texttt{width}, \texttt{height} (int): The dimensions of the output image in pixels.
        \item \texttt{maxiter} (int): The maximum number of iterations.
        \item \texttt{sample\_size} (int): The number of random samples to generate.
    \end{itemize}
    \textbf{Returns:}
    \begin{itemize}
        \item \texttt{numpy.ndarray}: A 4D histogram representing the Buhdabrot fractal.
    \end{itemize}
\end{enumerate}

\section{Creating the Animations}
Once I completed the functions, I began programming the Manim animations. Rather than following a strict order,
I worked on the scenes that made the most sense at the time, guided by a general vision of the final product.
I started by rendering a static Mandelbrot set and tracing the orbit of a point. To achieve this,
I used \texttt{ComplexValueTracker}s to animate the changing point and equation text,
while applying the inferno colormap for visualization.
\textbf{Here is the code for the animations}

\lstinputlisting[language=Python,firstline=202,lastline=276, breaklines]{Main.py}

The next scene I created highlighted the orbit of a single point in the Mandelbrot set on a \texttt{ComplexPlane}. 
I used a point cloud to visualize the orbit of the point and arrows to indicate transitions between points. 
Additionally, I utilized a \texttt{ComplexValueTracker} to animate the movement of the point and a \texttt{MathTex} 
object to dynamically display the equation representing the point.
\textbf{This is the code for the scene:}

\lstinputlisting[language=Python,firstline=129,lastline=199]{Main.py}

After animating the oribit of a single point, I deciding to animate the orbit of multiple points, once agian
using \texttt{ComplexValueTracker}s to animate the \texttt{Dot} and the \texttt{MathTex} objects, as well as \texttt{mandelbrot\_point\_cloud} function 
to generate the orbits. it was also, once agian on the background of a \texttt{ComplexPlane}.
\textbf{Here is the code for the scene:}

\lstinputlisting[language=Python,firstline=278,lastline=325, breaklines]{Main.py}

The next scene I worked on was a text-based scene that explained the concept of the Mandelbrot set.
I used a \texttt{MathTex} object to display the equations, and a \texttt{Text} object to display the text.
\textbf{The code for the scene is as follows:}

\lstinputlisting[language=Python,firstline=16,lastline=82, breaklines]{Main.py}

After that, I Made a scene explaining the concept of Julia sets. The scene uses the \texttt{julia} function
to generate the Julia set for a specific complex number. I used a \texttt{ComplexValueTracker} to animate
changing the value of c and the inferno colormap to color the image.
\textbf{Here is the code for the scene:}

\lstinputlisting[language=Python,firstline=329,lastline=407, breaklines]{Main.py}

Then I tried to make a scene explaining how the Mandelbrot set acts as a map for Julia sets.
I used the \texttt{julia} function and loops to generate the Julia sets for each point ina grid. this was very difficult.
Then, I used a updater to animate increasing the reslution of the grid while deacreasing the reslution of each julia set,
showing how the Mandelbrot set acts as a map for Julia sets. however, I was not able to get this to work properly, as my grid
logic was incorrect.
\textbf{Here is the code for the scene:}

\lstinputlisting[language=Python, firstline=409, lastline=465, breaklines]{Main.py}

The hardest scene I made explored some generalizations of the Mandelbrot set, including the Multibrots and Buhdagrams.
after countless hours of work and many rounds of debugging, I was unfortunately unable to get the scene to work properly.
I used the \texttt{multibrot} and \texttt{buhdabrot} functions to generate the images, and a \texttt{ValueTracker} to animate the changing exponent.
I used the inferno colormap to color the multibrot, a false color method the color the buhdagram.
\textbf{Here is the code for the scene:}
\lstinputlisting[language=Python, firstline=468, lastline=605, breaklines]{Main.py}

After that, for the next hardest scene, I made a zoom sequnce expoloring the boundry of the Mandelbrot set.
This scene used the \texttt{mandelbrot\_from\_center} function to generate the images, and a \texttt{ValueTracker} to animate the zoom level.
I used the inferno colormap to color the images, and a array of 5 points to zoom on. once agian, after wasting countless hours of work and many rounds of debugging,
I was unable to meet runtime constraints. it would have taken hundreds of hours to render the scene, so I had to scrap it.
\textbf{Here is the code for the scene:}

\lstinputlisting[language=Python, firstline=607, lastline=665, breaklines]{Main.py}

the final scene I made was a simple outro scene, which used a \texttt{Text} object to display the text.
\textbf{The code for the scene is as follows:}

\lstinputlisting[language=Python, firstline=667, lastline=691, breaklines]{Main.py}

\section{Code Reveal Program}

although the code and process is shown here, I wanted to create a smooth way to reveal the code in the video.
I wrote a new python file which breaks the code into 40-line sections, and uses a \texttt{Code} object to display the code.
I really wanted people to know how much work actaully goes into 5 minutes of (crappy) animation, Which is why I made 
this program. Unironically the code reveal animation eneded up being longer than the actual animation.
\textbf{Here is the code for the program:}

\lstinputlisting[language=Python, breaklines]{code_reveal.py}

\section{Challenges and debugging}
I encountered numerous challenges while working on this project.
The most difficult aspect was debugging the code and ensuring that the animations rendered properly.
I spent countless hours trying to figure out why certain sections weren’t functioning as intended, often rewriting large portions of the code multiple times.
Learning how to use Manim and its features was also a steep learning curve.
The documentation was oftentimes unclear, so I had to rely heavily on online forums and community support.
People often underestimate how difficult coding can be and how much time it takes to get things working correctly.
Imagine if Google Docs took ten minutes to load every time you wanted to edit a document, and if you made a mistake, you’d have to wait another ten minutes just to see if it worked.
And instead of a user-friendly interface, you had to write everything in a plain text file.
That’s what coding can feel like: time-consuming, frustrating, and mentally exhausting.

Sometimes it gets so bad that \textit{unironically} 40\% of your video doesn’t even exist because it was bugged—something I experienced firsthand.
I had to turn to Stack Overflow multiple times, once waiting a week just for a reply.
Unlike a human, the Python interpreter doesn’t understand your intentions; it only interprets the code you wrote.
A single typo or subtle logic error can break everything, and the error messages are often vague—or it might even fail silently.
To make things more challenging, my subject matter involved complex mathematics and abstract thinking, which much added more room for error.
And as if that weren’t enough, imagine that if your English were grammatically correct but not perfectly clear and concise, Google Docs would take 1,000 times longer to load
and you’d need to learn even more math just to speed things up.
That’s what programming felt like throughout this project.


\end{document}