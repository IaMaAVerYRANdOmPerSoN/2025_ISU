\documentclass[preview]{standalone}
\usepackage[english]{babel}
\usepackage{amsmath}
\usepackage{amssymb}
\begin{document}
\begin{align*}
f_(0.4859213874215113+0.463561238032147j)(z) = z^2 + 0.49 + 0.46i
\end{align*}
\end{document}